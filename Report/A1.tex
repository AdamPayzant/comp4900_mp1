%\documentclass{article}
%\usepackage{amsmath}
%\author{ Adam Payzant - 101082175 \\ Aidan Crowther - 100980915 \\ Alex Cornish  - 101053176}
%\title{COMP4900A - Assignment 1}

%\begin{document}
%\maketitle
%\newpage



%\end{document}

\documentclass{article}

% if you need to pass options to natbib, use, e.g.:
%     \PassOptionsToPackage{numbers, compress}{natbib}
% before loading neurips_2020

% ready for submission
% \usepackage{neurips_2020}

% to compile a preprint version, e.g., for submission to arXiv, add add the
% [preprint] option:
%     \usepackage[preprint]{neurips_2020}

% to compile a camera-ready version, add the [final] option, e.g.:
%     \usepackage[final]{neurips_2020}

% to avoid loading the natbib package, add option nonatbib:
     \usepackage[nonatbib]{A1}

\usepackage[utf8]{inputenc} % allow utf-8 input
\usepackage[T1]{fontenc}    % use 8-bit T1 fonts
\usepackage{hyperref}       % hyperlinks
\usepackage{url}            % simple URL typesetting
\usepackage{booktabs}       % professional-quality tables
\usepackage{amsfonts}       % blackboard math symbols
\usepackage{nicefrac}       % compact symbols for 1/2, etc.
\usepackage{microtype}      % microtypography

\title{Linear Regression for Binary Classification}

% The \author macro works with any number of authors. There are two commands
% used to separate the names and addresses of multiple authors: \And and \AND.
%
% Using \And between authors leaves it to LaTeX to determine where to break the
% lines. Using \AND forces a line break at that point. So, if LaTeX puts 3 of 4
% authors names on the first line, and the last on the second line, try using
% \AND instead of \And before the third author name.

\author{%
  Adam Payzant\\
  \texttt{101082175} \\
  \And
  Aidan Crowther\\
  \texttt{100980915} \\
  \And
  Alex Cornish\\
  \texttt{101053176} \\
  % examples of more authors
  % \And
  % Coauthor \\
  % Affiliation \\
  % Address \\
  % \texttt{email} \\
  % \AND
  % Coauthor \\
  % Affiliation \\
  % Address \\
  % \texttt{email} \\
  % \And
  % Coauthor \\
  % Affiliation \\
  % Address \\
  % \texttt{email} \\
  % \And
  % Coauthor \\
  % Affiliation \\
  % Address \\
  % \texttt{email} \\
}

\begin{document}

\maketitle

\begin{abstract}
This investigative assignment was performed with the intended purpose of exploring the implementation of linear classification and the effect of various variables on its performance.
The resultant logistical classifier is to then be modified through the use of hyperparameters as well as modifying the data on which it is to train so as to explore the impact of various
variables on the performance of the classifier. Parameters such as the learning rate and the number of iterations were observed to have a significant impact on the performance of the implemented classifier. In addition the elimination of largely biased and non-descriptive features was observed to have a significant impact on the performance and accuracy of the model by reducing the number of consequent calculations, as well as by removing noisy data from the classifier.
\end{abstract}

\section{Introduction}

The first task to be accomplished within this project is to perform some basic analysis of the data contained within the datasets supplied in order to observe patterns of feature distribution, 
and using this data isolate potential targets for isolation when performing classification using fewer features. Following this primary analysis stage, a linear classification model is to be constructed, utilizing logistic regression on the supplied data. This models performance is to then be observed by the metrics of classification speed and accuracy, and is to be altered through the use of various modification techniques to observe their effects on the measured performance. These include the modification of input features as previously mentioned, as well as the use of various hyperparameters for the model, such as the learning rate and number of iterations to train on each subset of the data. The impacts of these modifications were observed to have a significant impact on the models performance, such as the impact of a large learning rate on the dataset with more features resulting in worse performance while a larger learning rate resulted in improved performance for the dataset with fewer features. It was also observed that more iterations are generally better for training, however, this does not hold completely true, as training on the same dataset began to negatively impact impact the performance past a certain point. This phenomenon is believed to be related to the overfitting of the model on these individual datasets, resulting in poor general performance. This is all to be performed using two provided datasets for analysis, each with a different number of features and a binary classification, providing imformation on bankruptcy of certain individuals and the diagnosis of hepatitis.

\newpage

\section{Datasets}

Provided for this evaluation are two datasets with a variety of features and a resultant binary classification for each set of these features. The first of these datasets provides various statistics on and about patients who either were or were not diagnosed with hepatitis. These include metrics such as the patients age, their pre-existing conditions, and various measurements of their general health. This dataset consists of 19 features, with 143 different samples, and a general distribution of positive to negative diagnoses of 116 positive to 27 negative. This dataset is relativly poorly distributed, however it does allow for significant linking between strongly correlated features and their final classification.\\

The second dataset focuses on the occurence of bankruptcy among a group of individuals given a number of unspecified features. This dataset consists of 64 features, and a total of 453 samples. The distribution of bankruptcies in this dataset is far more balanced than in the hepatitis dataset, with 203 bankruptcies and 250 non-bankruptcies. This dataset unlike the hepatitis dataset features a more balanced distribution of its results, however, it is also notable in that there are a number of features that contain data that is completely unrelated to the outcome, or at worst completely skewed to one side in a way that would potentially have negative effects on the performance of the final model. To this end, feature manipulation is expected to have a significant impact on the performance of this model, by removing the seemingly irrelevant data the effect of feature manipulation can be observed suring this investigation.

\section{Results}

The linear classification model implemented for this report has performed surprisingly well in classifying for both datasets. Initial tests using small, static, hyperparameter values provided troubling results, with low accuracies; however, this was due to the use of fixed hyperparameters for the task of building a functional model, as these values allowed the model to iterate quickly. Upon constructing a model that appeared to be learning, despite its abysmal performance, the K-Fold process was then modified to run through multiple iterations using various values for the learning rate and number of iterations. Doing this it was evident how much of an impact each parameter had on the model. Itartions were tested running from 10 times per sample to 5000 times per sample, with major impacts on the runtime. Iterations were fast to run when below ~500, however runtime rapidly became unbearable when running over 5000 iterations. Thankfully, the runtime growth was linear in complexity, and as such didn't become absurdly long. The results of varying the iterations showed that 1000 iterations resulted in the best performance for both datasets; this was a surprising result due to the sample sizes between the datasets varying by a wide degree. Although the results for this paramter converged on the same number of iterations for both datasets, it is believed that the cause of this may be more related to insufficient variance between the number of iterations tested, as the next lowest was 500 and the next highest was 5000. These range samples were selected to minimize time spent waiting on results, however it may be warranted to further investigate this parameter to ideally learn data from each sample set. As expected, excessively small numbers of iterations resulted in poor accuracy due to underfitting the data, while excessively large numbers resulted in poor accuracy due to overfitting, where the model was specializing on the sample and not the general problem.\\

\newpage

The other parameter that was tested was the learning rate of the model; of which rates from 0.01 to 0.8 were tested. Unlike with the iterations, varying the learning rate did not affect runtime of the model during training, and as such was easier to test with larger variance in its values. This parameter has provided interesting results into the learning of the resultant linear classification model, as the learning rates that provided the best performance were not affected by the number of features as much as by the range of values within these features. It is now apparent as to why this would be the case, since the bias which is modified by these steps is directly applied to the input features. It was thought that the larger feature set of the bankruptcy dataset would result in larger step values performing better, however it has become apparent that the local minimums found for these samples are narrower than imagined. As a result of this testing, the results for the optimization of learning rate becomes intuitively clear with this further understanding. The rates upon which the model performed best for each dataset were 0.01 and 0.1 for the bankruptcy and hepatitis datasets respectively. As with the number of iterations, fixed values were tested for these learning rates, and as a result it may be possible that these values are not in fact ideal for the sample sets. Despite this, these values were able to provide very high accuracy for the datasets and as such are deemed to be sufficiently performant for our purposes.\\

On running this analysis of hyperparameter values, it was possible to achieve classification accuracies of 83.86\% on the bankruptcy dataset, using a step size of 0.01 over 1000 iterations per sample; and 86.57\% on the hepatitis dataset, using a step size of 0.1 over 1000 iterations. These accuracies are significantly higher than random for both datasets, and perform better than always predicting the same result even on the hepatitis dataset, which as was mentioned has hevily skewed results. This lends credence to the likelihood that on the hepatitis dataset, the model has not overfitted to always return the same result, where as for the bankruptcy dataset it is relatively clear that the model has not become skewed in its classification.\\

As there is a lower number of attributes in the dataset for hepatitis.csv, the changes made by pruning the dataset of columns that diminish the accuracy of the prediction can be seen more easily. This generally occurs when the distribution of the values of a column are similar regardless of if the class label is 0 or 1, meaning that the column itself does not contribute valuably to the prediction. For example, by removing 5 out of 19 columns from hepatitus (bilirubin ,albumin ,protime ,steroid ,and ascites), an improvement in the accuracy of predictions from around 83\% to 88.7\%, whilst this improvement is not too significant, any improvement in the resulting accuracy is good. However, the same cannot be said of bankruptcy.csv, with its 64 attributes, which makes improving the accuracy by the most logical method rather difficult. In this instance, the method used to improve the accuracy of predictions was to run the initialisation of the model, then remove one attribute at a time, perform the testing, and find the resulting accuracy of that set of testing with the missing attribute, and then move on to the next column. Once that round of testing was complete, the highest resulting accuracy was compared against the original accuracy and if it was deemed an improvement, that attribute was removed more permenantly and the process was repeated until no more accurate model than the current one could be found. A problem arose in this method when removing one attribute at a time yielded no improvement to the bankruptcy model with the accuracy of most attributes being removed being about the same, which would indicate that all of them are either equally important or equally unimportant to the model.\\

\section{Discussion and Conclusion}

The result of this investigation has been a successful linear classifier implementation, while also providing a strong learning opportunity into the function and operation of linear classifiers as well as logistic regression. The observations of particular note are the impact of paramaters and feature pruning on classifier accuracy and performance. It was observed that iterating over each sample has performed better when iterations do not exceed 10x the sample size, while iteration ranges equal to or less than the sample size show similar results. Surprisingly, the variance in accuracies between these ranges is relatively marginal, which would seem to imply diminishing returns with larger iteration counts, especially when considering the risk of overfitting in these situations.\\

As for learning rate, it has become apparent that learning rate is heavily dependant on the actual data contained within each feature set rather than on the features themselves. These results are also enforced by the observation of performance of the model when changing the learning rate. By far the largest variance observed in model performance has been as a result of varying the learning rate, with significant impact on final accuracy.\\

The usefulness of pruning seems to be heavily dependent on the data itself, mainly regarding the number of attributes or the values of the dataset itself as for some sets, such as hepatitis.csv, it is obvious using the method used which attributes when removed improve the accuracy of predictions by a noticable amount; however, for some datasets, such as bankruptcy.csv, such things are not feasible due to the number of attributes and the very limited impact that their absence has on the accuracy of the model. In retrospect, it might work better by starting with the attributes that when removed lower the accuracy of predictions and start by adding the other ones back in.\\

As a result of this investigation, we have come to learn more on the subject of linear classification, and the importance of features and their distribution on the final classifier performance. We have seen the impact of hyperparameters on the learning of a model, and have been directed to possible avenues of improvement for the future. This would include further analysis into iteration count optimization, as overfitting is the major risk of large numbers of iterations, while the actual poerformance increase from them is poor. As a result finding a more ideal ratio between the sample size and number of iterations could prove fruitful. Similarly, further investigation into optimizing the learning rate may be ideal, as it was determined that this parameter is directly affected by the data contained within the dataset, perhaps it may be possible to determine this value as a function of the range of values contained. It may also be an interesting concept to explore the use of a changing learning rate, allowing the gradient descent algorithm to better fit into narrow minima.\\

Furthermore, some experimentation into pruning to find better ways to consistently determine what attributes improve or worsen the accuracy of predictions would be interesting and especially useful in some cases where certain attributes cloudy the data and make accurate prediction more difficult.\\

\section{Statement of Contributions}

Alex Cornish: Ones and zeros table modifications, data pruning, k-fold implementation.

\section{Appendix}

\url{https://github.com/AdamPayzant/comp4900_mp1/tree/master/plots}


\end{document}
